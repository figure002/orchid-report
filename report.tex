\documentclass[review,3p,twocolumn]{elsarticle}
\setlength{\footskip}{30pt}

%% Use the option review to obtain double line spacing
%% \documentclass[authoryear,preprint,review,12pt]{elsarticle}

%% Use the options 1p,twocolumn; 3p; 3p,twocolumn; 5p; or 5p,twocolumn
%% for a journal layout:
%% \documentclass[final,1p,times]{elsarticle}
%% \documentclass[final,1p,times,twocolumn]{elsarticle}
%% \documentclass[final,3p,times]{elsarticle}
%% \documentclass[final,3p,times,twocolumn]{elsarticle}
%% \documentclass[final,5p,times]{elsarticle}
%% \documentclass[final,5p,times,twocolumn]{elsarticle}

%% For including figures, graphicx.sty has been loaded in
%% elsarticle.cls. If you prefer to use the old commands
%% please give \usepackage{epsfig}

\usepackage[utf8]{inputenc}
\usepackage[T1]{fontenc}
\usepackage{lmodern}
\usepackage{tgpagella}

% Extensive support for hypertext.
\usepackage{hyperref}

% Easy access to the Lorem Ipsum dummy text.
\usepackage{lipsum}

% Pro­vides both fore­ground (text, rules, etc.) and back­ground colour man­age­ment.
\usepackage{color}

% Enhanced support for graphics.
\usepackage{graphicx}

% Produces figures which text can flow around.
\usepackage{wrapfig}

% Customising captions in floating environments.
\usepackage{caption}

% Publication quality tables.
\usepackage{booktabs}

% Long tables.
\usepackage{longtable}

%% The amssymb package provides various useful mathematical symbols.
%\usepackage{amssymb}

%% The amsthm package provides extended theorem environments.
%\usepackage{amsthm}

%% The lineno packages adds line numbers. Start line numbering with
%% \begin{linenumbers}, end it with \end{linenumbers}. Or switch it on
%% for the whole article with \linenumbers.
%% \usepackage{lineno}

\journal{Journal Name}

% Configure hyperlink colors.
\hypersetup{
    colorlinks,
    citecolor=black,
    filecolor=black,
    linkcolor=black,
    urlcolor=black
}

% Configure caption display.
\captionsetup{margin=10pt,font=small,labelfont=bf,labelsep=period}

% Define where the images can be found.
\DeclareGraphicsExtensions{.pdf,.png,.jpg}
\graphicspath{{./images/}}

% Define some commands.
\input{db-stats.inc}

\begin{document}

\begin{frontmatter}

%% Title, authors and addresses

%% use the tnoteref command within \title for footnotes;
%% use the tnotetext command for the associated footnote;
%% use the fnref command within \author or \address for footnotes;
%% use the fntext command for the associated footnote;
%% use the corref command within \author for corresponding author footnotes;
%% use the cortext command for the associated footnote;
%% use the ead command for the email address,
%% and the form \ead[url] for the home page:
%% \title{Title\tnoteref{label1}}
%% \tnotetext[label1]{}
%% \author{Name\corref{cor1}\fnref{label2}}
%% \ead{email address}
%% \ead[url]{home page}
%% \fntext[label2]{}
%% \cortext[cor1]{}
%% \address{Address\fnref{label3}}
%% \fntext[label3]{}

\title{Automated identification of slipper orchids using image analysis and artificial neural networks}

%% use optional labels to link authors explicitly to addresses:
%% \author[label1,label2]{}
%% \address[label1]{}
%% \address[label2]{}


\author[nbc]{Serrano Pereira}
\author[nbc,hsl,lu]{Barbara Gravendeel}
\author[hsl]{Patrick Wijntjes}
\author[nbc]{Rutger Vos}
\address[nbc]{Naturalis Biodiversity Center, P.O. Box 9517, Leiden, The Netherlands}
\address[hsl]{University of Applied Sciences Leiden, P.O. Box 382, Leiden, The Netherlands}
\address[lu]{Institute Biology Leiden, Leiden University, P.O. Box 546, Leiden, The Netherlands}

% Dave Roberts co-author?

\begin{abstract}
\lipsum[1]
\end{abstract}

\begin{keyword}
%% keywords here, in the form: keyword \sep keyword

%% PACS codes here, in the form: \PACS code \sep code

%% MSC codes here, in the form: \MSC code \sep code
%% or \MSC[2008] code \sep code (2000 is the default)
computer vision \sep image feature extraction \sep pattern recognition \sep
artificial neural networks \sep slipper orchids
\end{keyword}

\end{frontmatter}

%% \linenumbers

%% main text
\section{Introduction}
\label{sect:introduction}

\lipsum[1-3]

\section{Materials and methods}
\label{sect:methods}

\subsection{Reference photo collection}

A collection of reference photos was compiled by searching the internet for images, mostly through Google Image searches. The correct genus, section, and species for each image was verified by specialists and referenced to the literature \citet{Cribb1998, Pridgeon1999, Frosch2012} and phylogenetic reconstructions based on molecular analyses \citep{Li2011, Chochai2012}. Every photo was subsequently uploaded to an account on the online image hosting service Flickr (\url{https://www.flickr.com}). On Flickr, each image was annotated with standardized tags in the formats \texttt{genus:name}, \texttt{section:name}, and \texttt{species:name}. A custom Python script utilizing the Flickr API was used to download the entire image collection with metadata from the Flickr account to a local computer, where images are stored in a directory hierarchy and image annotations are stored in a database (\ref{sect:meta-database}).

\subsection{Image feature extraction}

Several image feature extraction methods were implemented and released in the form of a Python package called ImgPheno (\url{https://github.com/naturalis/imgpheno}). This Python package depends on OpenCV (\url{http://opencv.org/}) for computer vision functionality and NumPy (\url{http://www.numpy.org/}) for array and matrix manipulation functionalities. Image preprocessing and enhancement methods were implemented as well. These include colour image enhancement methods developed by \citet{Naik2003}.

\subsection{Training}

Several artificial neural network training routines were developed and released as a separate Python package called NBClassify (\url{https://github.com/naturalis/nbclassify/nbclassify}).

\subsection{Cross-validation}

Stratified k-folds cross-validation was performed to estimate the accuracy of the classifiers. Species names were used to set the classes for the cross-validations. In order to include all species, the number of folds was set to 4 (k=4). Cross-validation was also performed with 10 folds, only including the species represented by at least k=10 photos.

Principal components analysis (PCA) was performed on the training data to assess the suitability of the features to differentiate the photos by genus, section, and species.

\section{Results}
\label{sect:results}

\subsection{Reference photo collection}

In total {\PhotoCount} photos for {\SpeciesCount} species were collected (table~\ref{tbl:photo-counts}). The collection contains photos for the genera \textit{Cypripedium}, \textit{Mexipedium}, \textit{Paphiopedilum}, \textit{Phragmipedium}, and \textit{Selenipedium}. With only 4 photos for the genera \textit{Mexipedium} and \textit{Selenipedium}, the collection is highly unbalanced.

\begin{table*}[t]\footnotesize
    \caption{The number of photos collected per genus, as well as the number of sections and species per genus represented by the photo collection. A total of {\PhotoCount} photos for {\SpeciesCount} species were collected.}
    \begin{center}
    \begin{tabular}{llll}
    \toprule
    \textbf{Genus} & \textbf{Section} & \textbf{Species} & \textbf{Photos} \\
    \midrule
    \input{table-taxa-summary.inc}
    \bottomrule
    \end{tabular}
    \end{center}
    \label{tbl:photo-counts}
\end{table*}

\subsection{Image feature extraction}

The \verb/color_bgr_means/ method from ImgPheno extracts the BGR colour intensities from a region of interest in an image. This feature has a parameter, \verb/bins/, which is used to set the number of horizontal and vertical sections in which to divide the image. The method returns for each section the mean intensity for each of the components of the BGR colour space (blue, green, and red) (figure~\ref{fig:bgr-means-plots}).

\begin{figure*}[t]
    \centering
    \includegraphics[width=\textwidth]{BGR_means_plots}
    \caption{Plots of the mean BGR colour intensities for an image of \textit{P. druryi}. The plots display the mean intensities for the horizontal and vertical bins respectively.}
    \label{fig:bgr-means-plots}
\end{figure*}

\subsection{Cross-validation}

Stratified k-folds cross-validation was performed to estimate the accuracy of the classifiers. The accuracy for genus classification was 75\%, but accuracy drops dramatically as section (52\%) and species (48\%) are also included in the classification (table~\ref{tbl:x-validation-results}). With 10 folds and including only those species for which at least 10 photos are collected, results in slightly better accuracies (table~\ref{tbl:x-validation-results}), but this can probably be attributed to the reduced number of sections and species.

Principal components analysis (PCA) with orthogonal rotation (varimax) was conducted on training training data for genus classification containing -1 to 1 scaled mean colour intensities for the BGR colour space with 20 horizontal and vertical bins, which translates to 120 features. The Kaiser-Meyer-Olkin measure verified the sampling adequacy for the analysis with an overall Measure of Sampling Adequacy (MSA) of 0.92, and all MSA values for individual items were >0.76, which is well above the acceptable limit of 0.5. Bartlett's test of sphericity $\chi^2 (7140) = 452642$, $p = 0$, indicated that correlations between items were sufficiently large for PCA. An initial PCA was run to obtain eigenvalues for each component in the data. The scree plot constructed from the eigenvalues showed inflexion that would justify retaining 3 components (explaining 65\% of the variance). Given the large sample size (N = 1134) and the convergence of the scree plot, 3 components were retained in the final analysis. The same PCA workflow was repeated for training data for section classification within the genus Paphiopedilum, and for species classification within the section Parvisepalum of genus Paphiopedilum. Figure~\ref{fig:pca-plots} shows the scatter plots for the principal components.

\begin{table*}[t]\footnotesize
    \caption{Results for stratified k-fold cross-validations. Cross-validation was performed on three taxonomic ranks: genus, section, and species. The results for genus/section and genus/section/species combine the results from their respective ranks.}
    \begin{center}
    \begin{tabular}{lp{3cm}p{3cm}}
    \toprule
    \textbf{Classification} & \textbf{Accuracy (k=4)} & \textbf{Accuracy (k=10)} \\
    \midrule
    genus                   & 75\%    & 81\% \\
    section                 & 52\%    & 60\% \\
    species                 & 48\%    & 56\% \\
    genus/section           & 41\%    & 49\% \\
    genus/section/species   & 20\%    & 27\% \\
    \bottomrule
    \end{tabular}
    \end{center}
    \label{tbl:x-validation-results}
\end{table*}

\section{Conclusions and discussion}
\label{sect:conclusion}

% Need more and better photo material.
% Better feature extraction methods.

\lipsum[1]

\section*{Acknowledgements}

% De namen van alle fotografen in de format J.H. Simpson.

\lipsum[1]

%% If you have bibdatabase file and want bibtex to generate the
%% bibitems, please use
%%
%%  \bibliographystyle{elsarticle-num}
%%  \bibliography{<your bibdatabase>}

%% else use the following coding to input the bibitems directly in the
%% TeX file.

\section*{References}

%% Numbered
%\bibliographystyle{model1-num-names}

%% Numbered without titles
%\bibliographystyle{model1a-num-names}

%% Harvard
%\bibliographystyle{model2-names.bst}\biboptions{authoryear}

%% Vancouver numbered
%\usepackage{numcompress}\bibliographystyle{model3-num-names}

%% Vancouver name/year
%\usepackage{numcompress}\bibliographystyle{model4-names}\biboptions{authoryear}

%% APA style
%\bibliographystyle{model5-names}\biboptions{authoryear}

%% AMA style
%\usepackage{numcompress}\bibliographystyle{model6-num-names}

%% `Elsevier LaTeX' style
%\bibliographystyle{elsarticle-num}

%% Custom APA style
%\bibliographystyle{model5-names}
\bibliographystyle{apa-custom}
\biboptions{authoryear}

\bibliography{references}

%% The Appendices part is started with the command \appendix;
%% appendix sections are then done as normal sections
\appendix
\onecolumn

\section{Metadata database}
\label{sect:meta-database}

\begin{figure*}[!h]
    \centering
    \includegraphics[width=\textwidth]{Meta_database_diagram}
    \caption{Diagram of the metadata database used for storing taxonomic information for a collection of reference photos. Diagram created with SchemaCrawler (\url{http://schemacrawler.sourceforge.net}).}
    \label{fig:meta-database}
\end{figure*}

\section{Reference photo collection}
\label{sect:reference-photo-collection}

\begin{footnotesize}
\begin{longtable}{llllll}
    \caption{Taxa represented by the reference photo collection.}
    \label{tbl:taxa-stats}
    \endfirsthead
        \caption*{\textbf{Table \ref{tbl:taxa-stats}.} (continued)}
        \\\textbf{Taxa} & \textbf{Photos} \\
        \midrule
    \endhead
        %\midrule
        \caption*{\footnotesize\textit{(continued on next page)}}
    \endfoot
        \bottomrule
    \endlastfoot

    \toprule
    \textbf{Taxa} & \textbf{Photos} \\
    \midrule
    \input{table-taxa.inc}
\end{longtable}
\end{footnotesize}

\clearpage
\section{Scatter plots for principal components analyses}
\label{sect:pca-plots}

\begin{figure}[!h]
    \minipage{\textwidth}
        \includegraphics[width=0.5\linewidth]{genus_pca_plot}
        \includegraphics[width=0.5\linewidth]{{{Cypripedium.section_pca_plot}}}
    \endminipage
    \par\vfill
    \minipage{\textwidth}
        \includegraphics[width=0.5\linewidth]{{{Paphiopedilum.section_pca_plot}}}
        \includegraphics[width=0.5\linewidth]{{{Paphiopedilum.Parvisepalum.species_pca_plot}}}
    \endminipage
    \caption{Scatter plots for the principal components analyses. The plots are created from training data for 1) genus classification, 2) section classification for the genus Paphiopedilum, and 3) species classification for the section Parvisepalum of genus Paphiopedilum. For all plots the first principal component (PC1) was plotted against the second (PC2).}
    \label{fig:pca-plots}
\end{figure}

\end{document}
